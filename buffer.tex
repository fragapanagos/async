\documentclass{article}
\usepackage{mystyle}

\title{Buffering}
\author{Sam Fok}

\begin{document}
\maketitle
%%%%%%%%%%%%%%%%%%%%%%%%%%%%%%%%%%%%%%%%%%%%%%%%%%%%%%%%%%%%%%%%%%%%%%%%%%%%%%%

%%%%%%%%%%%%%%%%%%%%%%%%%%%%%%%%%%%%%%%%%%%%%%%%%%%%%%%%%%%%%%%%%%%%%%%%%%%%%%%
\part{FIFOs}

Stands for first-in-first-out. Each node in the buffer chain follows

\begin{csp}
*[Y!(X?)]
\end{csp}

%%%%%%%%%%%%%%%%%%%%%%%%%%%%%%%%%%%%%%%%%%%%%%%%%%%%%%%%%%%%%%%%%%%%%%%%%%%%%%%
\section{PCFB}

Pre-Charge Full Buffer

\begin{hse}
*[[~ya];[x0->y0+[]x1->y1+];xa+;en-;([ya];y0-,y1-),([~x0&~x1];xa-);en+]
\end{hse}

\subsubsection*{PRS}

\begin{prs2}
en & ~ya & x0 -> y0+
~en & ya -> y0-

en & ~ya & x1 -> y1+
~en & ya -> y1-
\end{prs2}

\begin{prs2}
en & (y0 | y1) -> xa+
~en & ~x0 & ~x1 -> xa-

xa -> en-
~y0 & ~y1 & ~xa -> en+
\end{prs2}

\subsubsection*{CMOS-implementable PRS version 0}

\begin{prs2}
~_en & ~ya & ~_x0 -> y0+
_en & ya -> y0-

~_en & ~ya & ~_x1 -> y1+
_en & ya -> y1-
\end{prs2}

\begin{prs2}
y0 -> _y0-
~y0 -> _y0+

y1 -> _y1-
~y1 -> _y1+
\end{prs2}

\begin{prs2}
~_en & (~_y0 | ~_y1) -> xa+
_en & _x0 & _x1 -> xa-

xa -> _xa-
~xa -> _xa+
\end{prs2}

\begin{prs2}
xa -> en-
~y0 & ~y1 & ~xa -> en+

en -> _en-
~en -> _en+
\end{prs2}

\subsubsection*{CMOS-implementable PRS version 1}

\begin{prs2}
en & _ya & x0 -> _y0-
~en & ~_ya -> _y0+

en & _ya & x1 -> _y1-
~en & ~_ya -> _y1+
\end{prs2}

\begin{prs2}
_y0 -> y0-
~_y0 -> y0+

_y1 -> y1-
~_y1 -> y1+
\end{prs2}

\begin{prs2}
en & (y0 | y1) -> _xa-
~en & ~x0 & ~x1 -> _xa+

_xa -> xa-
~_xa -> xa+
\end{prs2}

\begin{prs2}
xa -> en-
~y0 & ~y1 & ~xa -> en+
\end{prs2}

%%%%%%%%%%%%%%%%%%%%%%%%%%%%%%%%%%%%%%%%%%%%%%%%%%%%%%%%%%%%%%%%%%%%%%%%%%%%%%%
\section{NB}

No Buffer

\begin{hse}
*[[x0->y0+[]x1->y1+];[ya];xa+;[~x0&~x1];y0-,y1-;[~ya];xa-]
\end{hse}

\subsubsection*{PRS}

\begin{prs2}
x0 -> y0+
~x0 -> y0-

x1 -> y1+
~x1 -> y1-
\end{prs2}

\begin{prs2}
ya -> xa+
~ya -> xa-
\end{prs2}

This is really just wires, but if you have to implement it...

\subsubsection*{CMOS-implementable PRS version 0}

\begin{prs2}
x0 -> _y0-
~x0 -> _y0+

x1 -> _y1-
~x1 -> _y1+
\end{prs2}

\begin{prs2}
~_ya -> xa+
~ya -> xa-
\end{prs2}

transistor count: 6

\subsubsection*{CMOS-implementable PRS version 1}

\begin{prs2}
~_x0 -> y0+
_x0 -> y0-

~_x1 -> y1+
_x1 -> y1-
\end{prs2}

\begin{prs2}
ya -> _xa-
~ya -> _xa+
\end{prs2}

transistor count: 6

Alternate versions 0 and 1.

%%%%%%%%%%%%%%%%%%%%%%%%%%%%%%%%%%%%%%%%%%%%%%%%%%%%%%%%%%%%%%%%%%%%%%%%%%%%%%%
\section{WCHB}

Weak-Condition Half Buffer

\begin{hse}
*[[~ya][x0->y0+[]x1->y1+];xa+;[ya][~x0&~x1];y0-,y1-;xa-]
\end{hse}

\subsubsection*{PRS}

\begin{prs2}
~ya & x0 -> y0+
ya & ~x0 -> y0-

~ya & x1 -> y1+
ya & ~x1 -> y1-
\end{prs2}

\begin{prs2}
y0 | y1 -> xa+
~y0 & ~y1 -> xa-
\end{prs2}

\subsubsection*{CMOS-implementable PRS version 0}

\begin{prs2}
_ya & x0 -> _y0-
~_ya & ~x0 -> _y0+

_ya & x1 -> _y1-
~_ya & ~x1 -> _y1+
\end{prs2}

\begin{prs2}
~_y0 | ~_y1 -> xa+
_y0 & _y1 -> xa-
\end{prs2}

transistor count: 20

\subsubsection*{CMOS-implementable PRS version 1}

\begin{prs2}
~ya & ~_x0 -> y0+
ya & _x0 -> y0-

~ya & ~_x1 -> y1+
ya & _x1 -> y1-
\end{prs2}

\begin{prs2}
y0 | y1 -> _xa-
~y0 & ~y1 -> _xa+
\end{prs2}

transistor count: 20

Alternate versions 0 and 1. These versions are a bit wonky with their use of
staticizers and take their output as the input to staticizers.

\subsubsection*{CMOS-implementable PRS version 2}

With more conventional use of staticizers,

\begin{prs2}
_ya & x0 -> _y0-
~_ya & ~x0 -> _y0+

_ya & x1 -> _y1-
~_ya & ~x1 -> _y1+
\end{prs2}

\begin{prs2}
_y0 -> y0-
~_y0 -> y0+

_y1 -> y1-
~_y1 -> y1+
\end{prs2}

\begin{prs2}
y0 | y1 -> _xa-
~y0 & ~y1 -> _xa+
\end{prs2}

transistor count: 20

%%%%%%%%%%%%%%%%%%%%%%%%%%%%%%%%%%%%%%%%%%%%%%%%%%%%%%%%%%%%%%%%%%%%%%%%%%%%%%%
\end{document}
