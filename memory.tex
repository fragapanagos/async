\documentclass{article}
\usepackage{mystyle}

\begin{document}
%%%%%%%%%%%%%%%%%%%%%%%%%%%%%%%%%%%%%%%%%%%%%%%%%%%%%%%%%%%%%%%%%%%%%%%%%%%%%%%

%%%%%%%%%%%%%%%%%%%%%%%%%%%%%%%%%%%%%%%%%%%%%%%%%%%%%%%%%%%%%%%%%%%%%%%%%%%%%%%
\section{Bit cell}

The basic unit of a memory is a bit cell. This circuit holds a single bit.
Takes in and stores dual-rail, 1 hot data.

BITCELL
\begin{csp}
*[D?u]
\end{csp}

\begin{hse}
*[[df->bt-;bf+[]dt->bf-;bt+];da+;[~df&~dt];da-]
\end{hse}

\begin{prs2}
bt -> bf-
~bt -> bf+

bf -> bt-
~bf -> bt+
\end{prs2}

\begin{prs2}
dt -> bf-
df -> bt-
\end{prs2}

\begin{prs2}
~dt & ~df -> _da+
bf & df | bt & dt -> _da-

_da -> da-
~_da -> da+
\end{prs2}

16 transistors

%%%%%%%%%%%%%%%%%%%%%%%%%%%%%%%%%%%%%%%%%%%%%%%%%%%%%%%%%%%%%%%%%%%%%%%%%%%%%%%
\section{Bit cell array}

I could arrange as a 4*8=32. \\
32 cells * 16 transistors / cell / 4 neurons = 128 transistors / neuron \\
512 transistors / 4 neurons = 128 transistors / neuron \\
Though 4*7=28 could also work as I just need 26 bits of configuration. \\
28 cells * 16 transistors / cell / 4 neurons = 112 transistors / neuron \\

If I assume no staticizer on the write acknowledge, it only costs 12 transistors.
32 cells * 12 transistors / cell / 4 neurons = 96 transistors / neuron \\
28 cells * 12 transistors / cell / 4 neurons = 84 transistors / neuron \\

How would the neurons be stimulated? How will the data be addressed?

%%%%%%%%%%%%%%%%%%%%%%%%%%%%%%%%%%%%%%%%%%%%%%%%%%%%%%%%%%%%%%%%%%%%%%%%%%%%%%%
\end{document}
