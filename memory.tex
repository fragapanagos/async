\documentclass{article}
\usepackage{mystyle}

\begin{document}
%%%%%%%%%%%%%%%%%%%%%%%%%%%%%%%%%%%%%%%%%%%%%%%%%%%%%%%%%%%%%%%%%%%%%%%%%%%%%%%

%%%%%%%%%%%%%%%%%%%%%%%%%%%%%%%%%%%%%%%%%%%%%%%%%%%%%%%%%%%%%%%%%%%%%%%%%%%%%%%
\section{Bit cell}

The basic unit of a memory is a bit cell. This circuit holds a single bit.
Takes in and stores dual-rail, 1 hot data.

BITCELL
\begin{csp}
*[D?u]
\end{csp}

\begin{hse}
*[[df->bt-;bf+[]dt->bf-;bt+];da+;[~df&~dt];da-]
\end{hse}

\begin{prs2}
bt -> bf-
~bt -> bf+

bf -> bt-
~bf -> bt+
\end{prs2}

\begin{prs2}
dt -> bf-
df -> bt-
\end{prs2}

\begin{prs2}
~dt & ~df -> _da+
bf & df | bt & dt -> _da-

_da -> da-
~_da -> da+
\end{prs2}

16 transistors

%%%%%%%%%%%%%%%%%%%%%%%%%%%%%%%%%%%%%%%%%%%%%%%%%%%%%%%%%%%%%%%%%%%%%%%%%%%%%%%
\section{Bit cell array (6T cells)}

BITCELL\_GATED
\begin{csp}
*[D?u]
\end{csp}

\begin{hse}
*[[we];[df->bt-;bf+[]dt->bf-;bt+];da+;[~df&~dt];da-]
\end{hse}

\begin{prs2}
bt -> bf-
~bt -> bf+

bf -> bt-
~bf -> bt+
\end{prs2}

\begin{prs2}
df -> bt_-
bt & we -> bt_+

dt -> bf_-
bf & we -> bf_+
\end{prs2}

\begin{prs2}
~bf_ & we -> bf-
~bt_ & we -> bt-
\end{prs2}

\begin{prs2}
~dt & ~df -> _da+
df & bf_ | dt & bt_ -> _da-

_da -> da-
~_da -> da+
\end{prs2}

6 transistors per core \\
8 transistors per data line pair \\
4 transistors to staticize the write acknowledge \\

Consider the case of 32 bits per 4 neurons. \\
32 cells * 1 core / cell * 6 transistors / core = 192 transistors \\

Organize as 8 write-enable lines and 4 data line pairs \\
8 transistors / data line pair * 4 data line pairs = 32 transistors \\
192 + 32 = 224 transistors \\
224 transistors / 4 neurons = \textbf{56 transitors / neuron} \\

Organize as 32 write-enable lines and 1 data line pairs \\
8 transistors / data line pair * 1 data line pairs = 8 transistors \\
192 + 8 = 200 transistors \\
200 transistors / 4 neurons = \textbf{50 transitors / neuron} \\

BITCELL\_GATED\_CTRL
\begin{csp}
*[Y!(X?),w+;(Y,w)\star\!X];
\end{csp}

\begin{hse}
*[[x0->y0+,we+;[wa];xa+;[~x0];y0-,we-;[~wa];xa-
  []x1->y1+,we+;[wa];xa+;[~x1];y1-,we-;[~wa];xa-
 ]]
\end{hse}

\begin{prs2}
x0 -> y0+
~x0 -> y0-

x1 -> y1+
~x1 -> y1-
\end{prs2}

\begin{prs2}
x0 | x1 -> we+
~x0 & ~x1 -> we-
\end{prs2}

\begin{prs2}
we & wa -> xa+
~we & ~wa -> xa-
\end{prs2}

%%%%%%%%%%%%%%%%%%%%%%%%%%%%%%%%%%%%%%%%%%%%%%%%%%%%%%%%%%%%%%%%%%%%%%%%%%%%%%%
\section{Bit cell array (8T cells)}

BITCELL\_GATED
\begin{csp}
*[D?u]
\end{csp}

\begin{hse}
*[[we];[df->bt-;bf+[]dt->bf-;bt+];da+;[~df&~dt];da-]
\end{hse}

\begin{prs2}
bt -> bf-
~bt -> bf+

bf -> bt-
~bf -> bt+
\end{prs2}

\begin{prs2}
dt & we -> bf-
df & we -> bt-
\end{prs2}

\begin{prs2}
~dt & ~df -> _da+
we & bf & df | we & bt & dt -> _da-

_da -> da-
~_da -> da+
\end{prs2}

8 transistors per core \\
8 transistors per data line pair \\
4 transistors to staticize the write acknowledge \\

Consider the case of 32 bits per 4 neurons. \\
32 cells * 1 core / cell * 8 transistors / core = 256 transistors \\

Organize as 8 write-enable lines and 4 data line pairs \\
8 transistors / data line pair * 4 data line pairs = 32 transistors \\
256 + 32 = 288 transistors \\
288 transistors / 4 neurons = \textbf{72 transitors / neuron} \\

Organize as 32 write-enable lines and 1 data line pairs \\
8 transistors / data line pair * 1 data line pairs = 8 transistors \\
256 + 8 = 264 transistors \\
264 transistors / 4 neurons = \textbf{66 transitors / neuron} \\

BITCELL\_GATED\_CTRL
\begin{csp}
*[Y!(X?),w+;(Y,w)\star\!X];
\end{csp}

\begin{hse}
*[[x0->y0+,we+;[wa];xa+;[~x0];y0-,we-;[~wa];xa-
  []x1->y1+,we+;[wa];xa+;[~x1];y1-,we-;[~wa];xa-
 ]]
\end{hse}

\begin{prs2}
x0 -> y0+
~x0 -> y0-

x1 -> y1+
~x1 -> y1-
\end{prs2}

\begin{prs2}
x0 | x1 -> we+
~x0 & ~x1 -> we-
\end{prs2}

\begin{prs2}
wa -> xa+
~we & ~wa -> xa-
\end{prs2}

%%%%%%%%%%%%%%%%%%%%%%%%%%%%%%%%%%%%%%%%%%%%%%%%%%%%%%%%%%%%%%%%%%%%%%%%%%%%%%%
\end{document}
