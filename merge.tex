\documentclass{article}
\usepackage{mystyle}

\begin{document}
%%%%%%%%%%%%%%%%%%%%%%%%%%%%%%%%%%%%%%%%%%%%%%%%%%%%%%%%%%%%%%%%%%%%%%%%%%%%%%%

This document explores merges. 

What are their implementation costs?

%%%%%%%%%%%%%%%%%%%%%%%%%%%%%%%%%%%%%%%%%%%%%%%%%%%%%%%%%%%%%%%%%%%%%%%%%%%%%%%
\section{controlled merge}

Inputs are not exclusive.
Control signal indicates which signal to pass.

\begin{csp}
*[S?s;
    [s=0->P!(C0?)
    []s=1->P!(C1?)
    ]
 ]
\end{csp}

%%%%%%%%%%%%%%%%%%%%%%%%%%%%%%%%%%%%%%%%%%%%%%%%%%%%%%%%%%%%
\subsection{aMx1ofN}

\begin{hse}
*[[s0&c00->p0+;[pa];c0a+;[~c00];p0-;[~pa];c0a-
  []s0&c01->p1+;[pa];c0a+;[~c01];p1-;[~pa];c0a-
  []s1&c10->p0+;[pa];c1a+;[~c00];p0-;[~pa];c1a-
  []s1&c11->p1+;[pa];c1a+;[~c01];p1-;[~pa];c1a-
 ]]
\end{hse}

I assume that the downphase of the S control signal will be acknowledged elsewhere

\begin{prs2}
s0 & c00 | s1 & c10 -> p0+
s0 & ~c00 | s1 & ~c10 -> p0-

s0 & c01 | s1 & c11 -> p1+
s0 & ~c01 | s1 & ~c11 -> p1-
\end{prs2}

\begin{prs2}
s0 & pa -> c0a+
~s0 | ~pa -> c0a-

s1 & pa -> c1a+
~s1 | ~pa -> c1a-
\end{prs2}

\noindent Radix 2 accounting: 

\begin{center}
    \begin{tabular}{|r|l|l|}
    \hline
    rule & transistor count & comments \\ \hline
    $p[0,1]$ & 24 & \\ \hline
    $c[0,1]a$ & 8 & \\ \hline
    \hline total & 32 & \\ \hline
    \end{tabular}
\end{center}

\noindent Radix 4 accounting: 

\begin{center}
    \begin{tabular}{|r|l|l|}
    \hline
    rule & transistor count & comments \\ \hline
    $p[0,1,2,3]$ & 32 & \\ \hline
    $c[0,1,2,3]a$ & 80 & \\ \hline
    \hline total & 112 & \\ \hline
    \end{tabular}
\end{center}

ick...

%%%%%%%%%%%%%%%%%%%%%%%%%%%%%%%%%%%%%%%%%%%%%%%%%%%%%%%%%%%%
\subsection{eMx1ofN}

\begin{hse}
*[[pe];
  [s0->c0e+;
    [c00->p0+;[~pe];c0e-;[~c00];p0-
    []c01->p1+;[~pe];c0e-;[~c01];p1-
    ]
  []s1->c1e+;
    [c10->p0+;[~pe];c1e-;[~c10];p0-
    []c11->p1+;[~pe];c1e-;[~c11];p1-
    ]
 ]]
\end{hse}

\begin{prs2}
pe & s0 -> c0e+
~pe | ~s0 -> c0e-

pe & s1 -> c1e+
~pe | ~s1 -> c1e-
\end{prs2}

\begin{prs2}
c00 | c10 -> p0+
~c00 & ~c10 -> p0-

c01 | c11 -> p1+
~c01 & ~c11 -> p1-
\end{prs2}

wow this is really nice. This merge knows exactly when to expect input from each child which makes the output rules combinational.

\noindent Radix 2 accounting: 

\begin{center}
    \begin{tabular}{|r|l|l|}
    \hline
    rule & transistor count & comments \\ \hline
    $c[0,1]e$ & 8 & \\ \hline
    $p[0,1]$ & 8 & \\ \hline
    \hline total & 16 & \\ \hline
    \end{tabular}
\end{center}

\noindent Radix 4 accounting: 

\begin{center}
    \begin{tabular}{|r|l|l|}
    \hline
    rule & transistor count & comments \\ \hline
    $c[0,1,2,3]e$ & 16 & \\ \hline
    $p[0,1,2,3]$ & 32 & \\ \hline
    \hline total & 48 & \\ \hline
    \end{tabular}
\end{center}

%%%%%%%%%%%%%%%%%%%%%%%%%%%%%%%%%%%%%%%%%%%%%%%%%%%%%%%%%%%%%%%%%%%%%%%%%%%%%%%
\section{mutually exclusive merge}

%%%%%%%%%%%%%%%%%%%%%%%%%%%%%%%%%%%%%%%%%%%%%%%%%%%%%%%%%%%%
\subsection{aMx1ofN}

\begin{hse}
*[[c00->p0+;[pa];c0a+;[~c00];p0-;[~pa];c0a-
  []c01->p1+;[pa];c0a+;[~c01];p1-;[~pa];c0a-
  []c10->p0+;[pa];c1a+;[~c00];p0-;[~pa];c1a-
  []c11->p1+;[pa];c1a+;[~c01];p1-;[~pa];c1a-
 ]]
\end{hse}

\begin{prs2}
c00 | c10 -> p0+
~c00 & ~c10 -> p0-

c01 | c11 -> p1+
~c01 & ~c11 -> p1-
\end{prs2}

\begin{prs2}
(c00 | c01) & pa -> c0a+
~pa -> c0a-

(c10 | c11) & pa -> c1a+
~pa -> c1a-
\end{prs2}

\noindent Radix 2 accounting: 

\begin{center}
    \begin{tabular}{|r|l|l|}
    \hline
    rule & transistor count & comments \\ \hline
    $p[0,1]$ & 8 & \\ \hline
    $c[0,1]a$ & 16 & \\ \hline
    \hline total & 24 & \\ \hline
    \end{tabular}
\end{center}

\noindent Radix 4 accounting: 

\begin{center}
    \begin{tabular}{|r|l|l|}
    \hline
    rule & transistor count & comments \\ \hline
    $p[0,1,2,3]$ & 32 & \\ \hline
    $c[0,1,2,3]a$ & 40 & \\ \hline
    \hline total & 72 & \\ \hline
    \end{tabular}
\end{center}
%%%%%%%%%%%%%%%%%%%%%%%%%%%%%%%%%%%%%%%%%%%%%%%%%%%%%%%%%%%%
\subsection{avMx1ofN}

\begin{hse}
*[[c00->p0+;[pa&c0v];c0a+;[~c00];p0-;[~pa];c0a-
  []c01->p1+;[pa&c0v];c0a+;[~c01];p1-;[~pa];c0a-
  []c10->p0+;[pa&c1v];c1a+;[~c00];p0-;[~pa];c1a-
  []c11->p1+;[pa&c1v];c1a+;[~c01];p1-;[~pa];c1a-
 ]]
\end{hse}

\begin{hse}
*[[c0v|c1v];pv+[~c0v&~c1v];pv-]
\end{hse}

\begin{prs2}
c00 | c10 -> p0+
~c00 & ~c10 -> p0-

c01 | c11 -> p1+
~c01 & ~c11 -> p1-
\end{prs2}

\begin{prs2}
c0v & pa -> c0a+
~pa -> c0a-

c1v & pa -> c1a+
~pa -> c1a-
\end{prs2}

\begin{prs2}
c0v | c1v -> pv+
~c0v & ~c1v -> pv-
\end{prs2}

\noindent Radix 2 accounting: 

\begin{center}
    \begin{tabular}{|r|l|l|}
    \hline
    rule & transistor count & comments \\ \hline
    $p[0,1]$ & 8 & \\ \hline
    $c[0,1]a$ & 14 & \\ \hline
    $pv$ & 4 & \\ \hline
    \hline total & 26 & \\ \hline
    \end{tabular}
\end{center}

\noindent Radix 4 accounting: 

\begin{center}
    \begin{tabular}{|r|l|l|}
    \hline
    rule & transistor count & comments \\ \hline
    $p[0,1,2,3]$ & 32 & \\ \hline
    $c[0,1,2,3]a$ & 28 & \\ \hline
    $pv$ & 8 & \\ \hline
    \hline total & 68 & \\ \hline
    \end{tabular}
\end{center}

%%%%%%%%%%%%%%%%%%%%%%%%%%%%%%%%%%%%%%%%%%%%%%%%%%%%%%%%%%%%
\subsection{eMx1ofN}

\begin{hse}
*[[pe];c0e+,c1e+;
  [c00->p0+;[~pe];c0e-;[~c00];p0-;
  []c01->p1+;[~pe];c0e-;[~c01];p1-;
  []c10->p0+;[~pe];c1e-;[~c00];p0-;
  []c11->p1+;[~pe];c1e-;[~c01];p1-;
 ]]
\end{hse}

\begin{prs2}
pe -> c0e+
~pe -> c0e-

pe -> c1e+
~pe -> c1e-
\end{prs2}

\begin{prs2}
(c00 | c10) & c0e & c1e -> p0+
~c00 & ~c10 -> p0-

(c01 | c11) & c0e & c1e -> p1+
~c01 & ~c11 -> p1-
\end{prs2}

\noindent Radix 2 accounting: 

\begin{center}
    \begin{tabular}{|r|l|l|}
    \hline
    rule & transistor count & comments \\ \hline
    $c[0,1]e$ & 4 & \\ \hline
    $p[0,1]$ & 20 & \\ \hline
    \hline total & 24 & \\ \hline
    \end{tabular}
\end{center}

\noindent Radix 4 accounting: 

\begin{center}
    \begin{tabular}{|r|l|l|}
    \hline
    rule & transistor count & comments \\ \hline
    $p[0,1,2,3]$ & 8 & \\ \hline
    $c[0,1,2,3]a$ & 64 & \\ \hline
    \hline total & 72 & \\ \hline
    \end{tabular}
\end{center}

%%%%%%%%%%%%%%%%%%%%%%%%%%%%%%%%%%%%%%%%%%%%%%%%%%%%%%%%%%%%
\subsection{evMx1ofN}

\begin{hse}
*[[pe];c0e+,c1e+;
  [c00->p0+;[~pe];c0e-;[~c00];p0-;
  []c01->p1+;[~pe];c0e-;[~c01];p1-;
  []c10->p0+;[~pe];c1e-;[~c00];p0-;
  []c11->p1+;[~pe];c1e-;[~c01];p1-;
 ]]
\end{hse}

\begin{hse}
*[[c0v|c1v];pv+[~c0v&~c1v];pv-]
\end{hse}

\begin{prs2}
pe -> c0e+
~pe -> c0e-

pe -> c1e+
~pe -> c1e-
\end{prs2}

\begin{prs2}
c00 & c0v | c10 & c1v -> p0+
~c00 & ~c10 -> p0-

c01 & c0v | c11 & c1v -> p1+
~c01 & ~c11 -> p1-
\end{prs2}

\begin{prs2}
c0v | c1v -> pv+
~c0v & ~c1v -> pv-
\end{prs2}

\noindent Radix 2 accounting: 

\begin{center}
    \begin{tabular}{|r|l|l|}
    \hline
    rule & transistor count & comments \\ \hline
    $c[0,1]e$ & 4 & \\ \hline
    $p[0,1]$ & 20 & \\ \hline
    \hline total & 24 & \\ \hline
    \end{tabular}
\end{center}

\noindent Radix 4 accounting: 

\begin{center}
    \begin{tabular}{|r|l|l|}
    \hline
    rule & transistor count & comments \\ \hline
    $p[0,1,2,3]$ & 8 & \\ \hline
    $c[0,1,2,3]a$ & 64 & \\ \hline
    \hline total & 72 & \\ \hline
    \end{tabular}
\end{center}

%%%%%%%%%%%%%%%%%%%%%%%%%%%%%%%%%%%%%%%%%%%%%%%%%%%%%%%%%%%%%%%%%%%%%%%%%%%%%%%
\end{document}
